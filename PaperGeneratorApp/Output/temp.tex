\documentclass{article}
\usepackage[a4paper, total={7in, 9in}]{geometry}
\usepackage{amsmath}
\usepackage{graphicx}
\graphicspath{{D:/Computer Science/Projects/QuestionPaper/LaTeXFiles/Images/}}
\usepackage{enumitem}
\usepackage{multicol}
\usepackage{adjustbox}
\usepackage{tasks}
\title{
    Questions\\
    \vspace{0.2cm}
    \large Physics % Adds another centered line below the main title
}
\author{Sanjeev\thanks{Academy Of Physics.}}
\date{}
\begin{document}
\maketitle
\begin{enumerate}
\item If the charge on a capacitor is doubled, the value of its capacitance \(C\) will be :
\begin{tasks}(2)
\task Doubled
\task Halved
\task Remains Unchanged
\task None of these
\end{tasks}
\item The maximum electric field that a dielectric medium of a capacitor can withstand without breakdown (of its insulating property) is called its :
\begin{tasks}(2)
\task \(Polarization\)
\task \(Capacitance\)
\task \(Dielectric \; strength\)
\task \(Dielectric \;constant\)
\end{tasks}
\item Consider a parallel plate capacitor of \(10\;\mu\)F (microfarad) with air filled in the gap between the plates. Now, exactly one-half of the space between the plates is filled with a dielectric of dielectric constant \(4\), as shown in the figure.The capacity of the capacitor changes to :

\includegraphics[width = 0.25\textwidth]{3.png}

\begin{tasks}(2)
\task \(25\;\mu\)F
\task \(20\;\mu\)F
\task \(40\;\mu\)F
\task \(5\;\mu\)F
\end{tasks}
\item Two large metal plates are placed parallel to each other. The inner surfaces of plates are charged by \(+ \sigma\) and \(- \sigma\) \(C/m^2\). The electric field between the plates and outside the plates :

\includegraphics[width = 0.25\textwidth]{4.png}

\begin{tasks}(2)
\task \( \sigma/2\varepsilon_o \;,\; \sigma/\varepsilon_o \)
\task \( \sigma/\varepsilon_o \;,\; 0 \)
\task \( \sigma/2\varepsilon_o \;,\; 0 \)
\task \( 2\sigma/\varepsilon_o \;,\; \sigma/\varepsilon_o \)
\end{tasks}
\item Two identical capacitors \(C_1\) and \(C_2\) (equal in value) are connected in series with a battery of \(e.m.f \; V_o\). They are fully charged. Now, a dielectric slab is inserted between the plates of \(C_2\). The potential diff. across \(C_1\) will :
\begin{tasks}(2)
\task Increase
\task Decrease
\task Remains Same
\task Depends on internal resistance of the cell
\end{tasks}
\item A smooth parabolic wire track lies in the vertical plane \((x-y \; plane)\). The shape of track is defined by the equation \( y = x^2/a \) (where \(a\) is constant). A bead of mass \(m\) which can slide freely on the wire track, is placed at the position \(A(a, a)\). The track is rotated with constant angular speed \( \omega\) about \(y-axis\), such that there is no relative slipping between the ring and the track. Then \( \omega\) is equal to :
\begin{tasks}(2)
\task \large \( \sqrt{g / a}\)
\task \large \(\sqrt{2g / a}\)
\task \large \(\sqrt{g / 2a}\)
\task \large \(\sqrt{ \sqrt{2} g / a}\)
\end{tasks}
\item A uniform circular ring of mass per unit length \( \lambda\) and radius \(R\) is rotating with angular velocity \( \omega\) about its own axis in a gravity free space. Tension in the ring is :
\begin{tasks}(2)
\task \(Zero\)
\task \(\lambda \;\omega^2 R^2 /2\)
\task \(\lambda \;\omega^2 R^2\)
\task \(2\lambda \;\omega^2 R^2\)
\end{tasks}
\item A chain of mass per unit length \( \lambda\) and length \(1.5\; m\) rests on a fixed smooth sphere of radius \(R = (2/\pi)\; m\)  \; such that end \(A\) of chain is at the top of sphere while the other end is hanging freely as shown. The chain is held stationary by a horizontal thread \(PA\). The tension in this thread is :

\includegraphics[width = 0.25\textwidth]{8.png}

\begin{tasks}(2)
\task \large \(\lambda g \displaystyle \biggl( \frac{1}{2} + \frac{2}{\pi}\biggr)\)
\task \large \(\lambda g \displaystyle \biggl( \frac{\pi}{2} + \frac{2}{\pi}\biggr)\)
\task \large \(\lambda g \displaystyle \biggl( \frac{2}{\pi}\biggr)\)
\task \large \(None \; of\; These\)
\end{tasks}
\item An ideal gas with adiabatic exponent \( \gamma = 2\) goes through a cycle as shown in figure, in which absolute temperature varies \( \tau = 4\) times. Find efficiency of this cycle :

\includegraphics[width = 0.25\textwidth]{9.png}

\begin{tasks}(2)
\task \(\dfrac{1}{9}\) \vspace{0.2 cm}
\task \(\dfrac{1}{8}\)
\task \(\dfrac{1}{6}\)
\task \(\dfrac{1}{5}\)
\end{tasks}
\item The potential energy of a \(4\;kg\) particle free to move along the \(x-axis\) is given by \(U(x)= \displaystyle  \frac {x^ {3}}{3} - \frac {5x^ {2}}{2}  +6x +3\). Total mechanical energy of the particle is \(17 J\). Then the maximum kinetic energy is :
\begin{tasks}(2)
\task \(10\;J\)
\task \(2\;J\)
\task \(9.5\;J\)
\task \(0.5\;J\)
\end{tasks}
\item If a mercury droplet of radius \(R\) and surface tension \(S\) is broken into 8 smaller droplets of equal size. Then the work done by the external agency is :
\begin{tasks}(2)
\task \( \dfrac{4}{3} \pi R^3 S \)
\task \(\pi R^2 S \)
\task \(8\pi R^2 S \)
\task \( 4\pi R^2 S \)
\end{tasks}
\item If \(n\) drops of a liquid, each with surface energy \(E\), join to form a single drop then :
\begin{tasks}(2)
\task energy released in the process will be \(E(n-n^{1/3})\)
\task energy absorbed in the process will be \(E(n-n^{1/3})\)
\task energy released in the process will be \(E(n-n^{2/3})\)
\task energy absorbed in the process will be \(E(n-n^{2/3})\)
\end{tasks}
\item A soap bubble of radius \(R\) is surrounded by another soap bubble of radius \(2R\), as shown. If surface tension = \(S\), then the pressure inside the smaller soap bubble, in excess of the atmospheric pressure, will be :

\includegraphics[width = 0.25\textwidth]{13.png}

\begin{tasks}(2)
\task \(4S/R\)
\task \(3S/R\)
\task \(6S/R\)
\task \(None \; of \; these\)
\end{tasks}
\end{enumerate}
\end{document}
