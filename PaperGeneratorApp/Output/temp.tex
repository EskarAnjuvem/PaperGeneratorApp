\documentclass{article}
\usepackage[a4paper, total={7in, 9in}]{geometry}
\usepackage{amsmath}
\usepackage{graphicx}
\graphicspath{{D:/Computer Science/Projects/QuestionPaper/LaTeXFiles/Images/}}
\usepackage{enumitem}
\usepackage{multicol}
\usepackage{adjustbox}
\usepackage{tasks}
\title{
    Questions\\
    \vspace{0.2cm}
    \large Physics % Adds another centered line below the main title
}
\author{Sanjeev\thanks{Academy Of Physics.}}
\date{}
\begin{document}
\maketitle
\begin{enumerate}
\item If the charge on a capacitor is doubled, the value of its capacitance \(C\) will be :
\begin{tasks}(2)
\task Doubled
\task Halved
\task Remains Unchanged
\task None of these
\end{tasks}
\item The maximum electric field that a dielectric medium of a capacitor can withstand without breakdown (of its insulating property) is called its :
\begin{tasks}(2)
\task \(Polarization\)
\task \(Capacitance\)
\task \(Dielectric \; strength\)
\task \(Dielectric \;constant\)
\end{tasks}
\item Consider a parallel plate capacitor of \(10\;\mu\)F (microfarad) with air filled in the gap between the plates. Now, exactly one-half of the space between the plates is filled with a dielectric of dielectric constant \(4\), as shown in the figure.The capacity of the capacitor changes to :

\includegraphics[width = 0.25\textwidth]{3.png}

\begin{tasks}(2)
\task \(25\;\mu\)F
\task \(20\;\mu\)F
\task \(40\;\mu\)F
\task \(5\;\mu\)F
\end{tasks}
\item Two large metal plates are placed parallel to each other. The inner surfaces of plates are charged by \(+ \sigma\) and \(- \sigma\) \(C/m^2\). The electric field between the plates and outside the plates :

\includegraphics[width = 0.25\textwidth]{4.png}

\begin{tasks}(2)
\task \( \sigma/2\varepsilon_o \;,\; \sigma/\varepsilon_o \)
\task \( \sigma/\varepsilon_o \;,\; 0 \)
\task \( \sigma/2\varepsilon_o \;,\; 0 \)
\task \( 2\sigma/\varepsilon_o \;,\; \sigma/\varepsilon_o \)
\end{tasks}
\item Two identical capacitors \(C_1\) and \(C_2\) (equal in value) are connected in series with a battery of \(e.m.f \; V_o\). They are fully charged. Now, a dielectric slab is inserted between the plates of \(C_2\). The potential diff. across \(C_1\) will :
\begin{tasks}(2)
\task Increase
\task Decrease
\task Remains Same
\task Depends on internal resistance of the cell
\end{tasks}
\item A smooth parabolic wire track lies in the vertical plane \((x-y \; plane)\). The shape of track is defined by the equation \( y = x^2/a \) (where \(a\) is constant). A bead of mass \(m\) which can slide freely on the wire track, is placed at the position \(A(a, a)\). The track is rotated with constant angular speed \( \omega\) about \(y-axis\), such that there is no relative slipping between the ring and the track. Then \( \omega\) is equal to :
\begin{tasks}(2)
\task \large \( \sqrt{g / a}\)
\task \large \(\sqrt{2g / a}\)
\task \large \(\sqrt{g / 2a}\)
\task \large \(\sqrt{ \sqrt{2} g / a}\)
\end{tasks}
\item A uniform circular ring of mass per unit length \( \lambda\) and radius \(R\) is rotating with angular velocity \( \omega\) about its own axis in a gravity free space. Tension in the ring is :
\begin{tasks}(2)
\task \(Zero\)
\task \(\lambda \;\omega^2 R^2 /2\)
\task \(\lambda \;\omega^2 R^2\)
\task \(2\lambda \;\omega^2 R^2\)
\end{tasks}
\item A chain of mass per unit length \( \lambda\) and length \(1.5\; m\) rests on a fixed smooth sphere of radius \(R = (2/\pi)\; m\)  \; such that end \(A\) of chain is at the top of sphere while the other end is hanging freely as shown. The chain is held stationary by a horizontal thread \(PA\). The tension in this thread is :

\includegraphics[width = 0.25\textwidth]{8.png}

\begin{tasks}(2)
\task \large \(\lambda g \displaystyle \biggl( \frac{1}{2} + \frac{2}{\pi}\biggr)\)
\task \large \(\lambda g \displaystyle \biggl( \frac{\pi}{2} + \frac{2}{\pi}\biggr)\)
\task \large \(\lambda g \displaystyle \biggl( \frac{2}{\pi}\biggr)\)
\task \large \(None \; of\; These\)
\end{tasks}
\item An ideal gas with adiabatic exponent \( \gamma = 2\) goes through a cycle as shown in figure, in which absolute temperature varies \( \tau = 4\) times. Find efficiency of this cycle :

\includegraphics[width = 0.25\textwidth]{9.png}

\begin{tasks}(2)
\task \(\dfrac{1}{9}\) \vspace{0.2 cm}
\task \(\dfrac{1}{8}\)
\task \(\dfrac{1}{6}\)
\task \(\dfrac{1}{5}\)
\end{tasks}
\item The potential energy of a \(4\;kg\) particle free to move along the \(x-axis\) is given by \(U(x)= \displaystyle  \frac {x^ {3}}{3} - \frac {5x^ {2}}{2}  +6x +3\). Total mechanical energy of the particle is \(17 J\). Then the maximum kinetic energy is :
\begin{tasks}(2)
\task \(10\;J\)
\task \(2\;J\)
\task \(9.5\;J\)
\task \(0.5\;J\)
\end{tasks}
\item If a mercury droplet of radius \(R\) and surface tension \(S\) is broken into 8 smaller droplets of equal size. Then the work done by the external agency is :
\begin{tasks}(2)
\task \( \dfrac{4}{3} \pi R^3 S \)
\task \(\pi R^2 S \)
\task \(8\pi R^2 S \)
\task \( 4\pi R^2 S \)
\end{tasks}
\item If \(n\) drops of a liquid, each with surface energy \(E\), join to form a single drop then :
\begin{tasks}(2)
\task energy released in the process will be \(E(n-n^{1/3})\)
\task energy absorbed in the process will be \(E(n-n^{1/3})\)
\task energy released in the process will be \(E(n-n^{2/3})\)
\task energy absorbed in the process will be \(E(n-n^{2/3})\)
\end{tasks}
\item A soap bubble of radius \(R\) is surrounded by another soap bubble of radius \(2R\), as shown. If surface tension = \(S\), then the pressure inside the smaller soap bubble, in excess of the atmospheric pressure, will be :

\includegraphics[width = 0.25\textwidth]{13.png}

\begin{tasks}(2)
\task \(4S/R\)
\task \(3S/R\)
\task \(6S/R\)
\task \(None \; of \; these\)
\end{tasks}
\item Two soap bubbles of radii \(R_1\) and \(R_2\) equal to \(4\; cm\) and \(5\;cm\) are touching each other over a common surface \(S_1S_2\) (shown in figure). Radius of the common surface will be :

\includegraphics[width = 0.25\textwidth]{14.png}

\begin{tasks}(2)
\task \(4\;cm\)
\task \(20\;cm\)
\task \(5\;cm\)
\task \(4.5\; cm\)
\end{tasks}
\item Graph between the mass of liquid inside the capillary and the radius of capillary :
\begin{tasks}(2)
\task \begin{minipage} {0.25\textwidth} \includegraphics[width=0.8\textwidth]{15A.png}\end{minipage}
\task \begin{minipage} {0.25\textwidth} \includegraphics[width=0.8\textwidth]{15B.png}\end{minipage}
\task \begin{minipage} {0.25\textwidth} \includegraphics[width=0.8\textwidth]{15C.png}\end{minipage}
\task \begin{minipage} {0.25\textwidth} \includegraphics[width=0.8\textwidth]{15D.png}\end{minipage}
\end{tasks}
\item A particle of mass \(2\;kg\) moves in a straight line. If \(v\) is the velocity at a distance \(x\) from a fixed point on the line and \(v^2 = 3 - 4 x^2\) , then :
\begin{tasks}(2)
\task The motion continues along \(+x\)-direction only
\task The graph of \(v\) versus \(x\) would be a straight line
\task The angular frequency of oscillation is \(4\) rad/s
\task The total energy of oscillation is \(3\;J\)
\end{tasks}
\item A particle moves along the \(z-axis\) according to the equation \(z=5+12\;cos\biggl(2\pi t + \displaystyle \frac{\pi}{2}\biggl)\), where \(z\) is in \(cm\) and \(t\) is in seconds. Select the correct alternative :
\begin{tasks}(2)
\task The motion of the particle is SHM with mean position at \(z = 12\;cm\)
\task The motion of the particle is SHM with one extreme position as \(-7\;cm\)
\task The motion of the particle is Oscillatory but not SHM
\task Amplitude of SHM is \(17\;cm\)
\end{tasks}
\item The amplitude of a particle in SHM is \(5\;cm\) and its time period is \(\pi\). At a displacement of \(3\;cm\) from its mean position the velocity in \(cm/sec\) will be :
\begin{tasks}(2)
\task 8
\task 12
\task 2
\task 16
\end{tasks}
\item At a particular position the velocity of a particle in SHM with amplitude \(A\) is \(\sqrt{3}/2\;\) of that at its mean position. In this position, its displacement is :
\begin{tasks}(2)
\task \(A/2\)
\task \(\sqrt{3}A/2\)
\task \(A/\sqrt{2}\)
\task \(\sqrt{2}A\)
\end{tasks}
\item If the maximum velocity of a particle in SHM is \(v_o\), then its velocity at half the amplitude from position of rest will be :
\begin{tasks}(2)
\task \(v_o/2\)
\task \(v_o\)
\task \(v_o \sqrt{3/2}\)
\task \(v_o\sqrt{3}/2\)
\end{tasks}
\item The velocities of a particle executing S.H.M. are \(30\) cm/s and \(16\) cm/s when its displacements are \(8\) cm and \(15\) cm from the equilibrium position. Then its amplitude of oscillation in cm is :
\begin{tasks}(2)
\task 25
\task 21
\task 17
\task 13
\end{tasks}
\item The figure shows two coherent microwave sources \(S_1\) and \(S_2\) emitting waves of wavelength \(\lambda\) and separated by a distance \(3\lambda\) . The minimum non-zero value of \(y\) for point \(P\) to be an intensity maximum is (\(D>>\lambda\)) :

\includegraphics[width = 0.25\textwidth]{22.png}

\begin{tasks}(2)
\task \(D\)
\task \(\sqrt{3}D\)
\task \(\sqrt{5}D/2\)
\task \(2\sqrt{2}D\)
\end{tasks}
\item A block of mass m is connected to three springs as shown in the figure. The block is displaced down slightly, from its mean position. The time period of oscillation is :

\includegraphics[width = 0.25\textwidth]{23.png}

\begin{tasks}(2)
\task \( \displaystyle 2\pi \sqrt{\frac{m}{k}} \)
\task \(\displaystyle \pi \sqrt{\frac{m}{k}} \)
\task \( \displaystyle2\pi \sqrt{\frac{m}{2k}} \)
\task \( \displaystyle4\pi \sqrt{\frac{m}{k}} \)
\end{tasks}
\item In the figure shown, a spring fixed to wall is initially undeformed. Now it is compressed by \(\sqrt{2}A\). Consider the collision of the block with the wall to be perfectly elastic. The time period of oscillation is :

\includegraphics[width = 0.25\textwidth]{24.png}

\begin{tasks}(2)
\task \(\displaystyle 2\pi \sqrt{\frac{m}{k}} \)
\task \(\displaystyle 4\pi \sqrt{\frac{m}{k}} \)
\task \(\displaystyle \frac{3\pi}{2} \sqrt{\frac{m}{k}} \)
\task \(\displaystyle \frac{5\pi}{2} \sqrt{\frac{m}{k}} \)
\end{tasks}
\item Three identical rods of mass \(M\) and length \(L\)  are placed on one another on the table so as to produce the maximum overhang as shown in figure. The maximum possible overhang will be :

\includegraphics[width = 0.25\textwidth]{25.png}

\begin{tasks}(2)
\task \(\displaystyle \frac{2L}{3} \)
\task \(\displaystyle \frac{11L}{12} \)
\task \(\displaystyle \frac{L}{3} \)
\task \(\displaystyle \frac{13L}{12} \)
\end{tasks}
\item As shown in figure a small block of mass \(m = 1\; kg\) is placed over a wedge of mass \(M = 4\; kg\). All surfaces are smooth. Mass \(m\) is released from rest from top position of wedge. Find the velocity of block at the instant, it is leaving the wedge :

\includegraphics[width = 0.25\textwidth]{26.png}

\begin{tasks}(2)
\task \(\sqrt{2}\)
\task \(4 \)
\task \(4\sqrt{2}\)
\task \(2 \)
\end{tasks}
\item A solid sphere of mass \(2\;kg\) and radius \(1\;m\) is moving on a smooth ground with linear velocity \(v_o = 4\; m/s\) and angular velocity \(\omega_o=9 \;rad/s\) as shown in figure. It collides elastically with a rough wall of coefficient of friction \(\mu\) and after collision with the wall rolls without slipping in opposite direction. Find the coefficient of friction \(\mu\) :

\includegraphics[width = 0.25\textwidth]{27.png}

\begin{tasks}(2)
\task \(\displaystyle\frac{3}{4}\)
\task \(\displaystyle\frac{3}{5}\)
\task \(\displaystyle\frac{1}{4}\)
\task \(\displaystyle\frac{1}{2} \)
\end{tasks}
\item The disc rolls without slipping on a smooth horizontal surface as shown. The speed of the centre \(C\) is \(v\). The speed of point \(A\) on the disc at the instant shown is :

\includegraphics[width = 0.25\textwidth]{28.png}

\begin{tasks}(2)
\task \(v\)
\task \(v\cos\theta\)
\task \(\sqrt{2}v\)
\task \(v\sec\theta \)
\end{tasks}
\item A small sphere of mass \( 1 \) kg is rolling without slipping with linear speed \( v =\displaystyle \sqrt{\frac{200}{7}} \text{ m/s}.\) It leaves the inclined plane at point \( C \). Find the kinetic energy at the top just before leaving the inclined plane, neglecting any impact at \( B \) and assuming no slipping anywhere :

\includegraphics[width = 0.25\textwidth]{29.png}

\begin{tasks}(2)
\task \(20\;J\)
\task \(100/7\;J\)
\task \(10\;J\)
\task \(15\;J\)
\end{tasks}
\item Initial angular velocity of the system shown is \(\omega_o\) and frame length is \(a\) . System is rotating about the vertical axle and frame has negligible mass compared to the four point masses each of mass \(m\). Due to an internal mechanism the spokes in the frame lengthen to \(2a\). Find new angular velocity of the system :

\includegraphics[width = 0.25\textwidth]{30.png}

\begin{tasks}(2)
\task \( \displaystyle\omega_0 \)
\task \( \displaystyle\frac{3\omega_0}{4} \)
\task \( \displaystyle\frac{\omega_0}{2} \)
\task \( \displaystyle\frac{\omega_0}{4} \)
\end{tasks}
\item A current in circuit is given by \( i = 3 + 4 \sin \omega t \). Then the effective value of current is :
\begin{tasks}(2)
\task \( 5 \)
\task \( \sqrt{7} \)
\task \( \sqrt{17} \)
\task \( \sqrt{10} \)
\end{tasks}
\item The incorrect statement is :
\begin{tasks}(2)
\task A.C. meters can be used to measure D.C
\task D.C meters cannot be used to measure A.C
\task A.C. and D.C. meters are based on the heating effect of current
\task A.C. meter reads rms value of current
\end{tasks}
\item If \( I_1, I_2, I_3 \) and \( I_4 \) are the respective r.m.s. values of the time-varying currents as shown in the four cases I, II, III, and IV. Then identify the correct relations :

\includegraphics[width = 1\textwidth]{33.png}

\begin{tasks}(2)
\task \( I_1 = I_2 = I_3 = I_4 \)
\task \( I_3 > I_1 = I_2 > I_4 \)
\task \( I_3 > I_4 > I_2 = I_1 \)
\task \( I_3 > I_2 > I_1 > I_4 \)
\end{tasks}
\item The effective value of current \(  i = 2 \sin 100 \pi t + 2 \sin(100 \pi t + 30^\circ) \) is :
\begin{tasks}(2)
\task \( \sqrt{2} \) A
\task \( 2\sqrt{2} + \sqrt{3} \) A
\task \( 4 \) A
\task None
\end{tasks}
\item If \( I = 2\sqrt{t} \) ampere then calculate rms values over \( t = 2 \) to \( 4 \) s :
\begin{tasks}(2)
\task \( \displaystyle \frac{\sqrt{3}}{2} \)
\task \( 2\sqrt{3} \)
\task \( \sqrt{3} \)
\task \( 4\sqrt{3} \)
\end{tasks}
\item In an AC circuit an alternating voltage \( \varepsilon = 200\sqrt{2} \sin 100 t \) volts is connected to a capacitor of capacity \( 1\mu F \). The r.m.s. value of the current in the circuit is :
\begin{tasks}(2)
\task 10 mA
\task 100 mA
\task 200 mA
\task 20 mA
\end{tasks}
\item In an AC circuit containing a pure capacitor, across which an AC emf \( \varepsilon = 100 \sin(1000t) \) volt is applied. If the peak value of the current is 200 mA, then the value of the capacitor is :
\begin{tasks}(2)
\task 2 \(\mu F\)
\task 20 \(\mu F\)
\task 5 \(\mu F\)
\task 500 \(\mu F\)
\end{tasks}
\item A student connects a long air cored – coil of manganin wire to a 100 V D.C. supply and records a current of 25 amp. When the same coil is connected across 100 V, 50 Hz a.c. the current reduces to 20 A. The reactance of the coil is :
\begin{tasks}(2)
\task \( 4 \;\Omega \)
\task \( 3 \;\Omega \)
\task \( 5 \;\Omega \)
\task None
\end{tasks}
\item In a purely inductive circuit, the applied voltage \( V = 50\sqrt{2} \sin (100\;\pi t) \) volt and ammeter reading is 2A then calculate value of \( L \) :
\begin{tasks}(2)
\task \( \displaystyle \frac{1}{2\pi} \;H \)
\task \( \displaystyle \frac{1}{4\pi} \;H \)
\task \( \displaystyle \frac{1}{\pi} \;H \)
\task None
\end{tasks}
\item If the power factor of an R-L series circuit is \( \frac{1}{2} \) when applied voltage is \( V = 100 \sin (100\pi t) \) volt and resistance of circuit is \( 200\;\Omega \), then calculate the inductance of the circuit :
\begin{tasks}(2)
\task \( \displaystyle 2\sqrt{3}\pi \;H \)
\task \( \displaystyle \frac{2\sqrt{3}}{\pi} \;H \)
\task \( \displaystyle \frac{\pi}{2\sqrt{3}} \;H \)
\task \( \displaystyle \frac{\sqrt{3}}{2\pi} \;H \)
\end{tasks}
\end{enumerate}
\end{document}
